\input texinfo
@settitle Perceptron Manual

@copying
This manual is for MLP project.

Copyright @copyright{} 2022 bperegri irobin
@end copying

@titlepage
@title Perceptron Manual
@author by bperegri, irobin
@page
@vskip 0pt plus 1filll
@insertcopying
@end titlepage

@contents

@node Top

@menu
* General principles : GeneralPrinciples. What is this all about?
* Learning ::                             How to train a neural network.
* Making predictions : MakingPredictions. It works!
* Tests ::                                How good is it?
* Save and Load : SaveLoad.               How not to waste time.
* Research ::                             How to waste a lot of time.
@end menu

@node GeneralPrinciples, Learning, Top, Top
@chapter General principles
  This project is meant to represent posibilities of using neural networks and machine learning and different aspects of those.

  In this particular example a neural network is used to classify written letters. Users can chose between two different implementations of neural networks (matrix and graph) and number of internal layers (2- 5).

  It is possible to train a network from scratch or load from file and continue training, though the process might take a while. The result of training is a combination of weights for signals between neurons of the network. Once trained, the network is able to discern different letters of latin alphabet input through an integrated drawing field or loaded as a .bmp file.

@node Learning, MakingPredictions, GeneralPrinciples, Top
@chapter Learning
  To start learning process you need to chose a train file (2nd button in the top row), chose number of epochs (next to "Start Learning" button) and push the "Start Learning" button. As an option you can chose the cross-validation mode of learning (checkbox over "Start Learning") and k - coefficient (5 or 10) of proportion. Cross-validation takes about 4 - 8 times longer than one epoch of learning. In a dialoge window you'll need to chose between continuing process with current weights or to start over with random weights.

  If you chose cross-validation mode or if a test file is chosen, you will recieve a report in the end of learning process, with diagram showing proportion of wrong predictions changing over the epochs.

@node MakingPredictions, Tests, Learning, Top
@chapter Making predictions
  To see a prediction of a network you need to input a written letter. You can draw one in an integrated drawing field - hold left mouse button to draw a line, click right mouse button to erase everything. After each line drawn the chosen network will try to classify an input letter. The prediction will appear in a field in the top right.

  Also, you can load an image from a .bmp file. The prediction will also appear in the top right.

  Remember, that for those predictions to be accurate you must use a trained network (trained by yourself or with loaded weights, git repository contains a set of weights files trained for average accuracy no less than 0.8 on a test file you can also find in repository).

@node Tests, SaveLoad, MakingPredictions, Top
@chapter Tests
  You can start a test process for the chosen network (type and number of layers) via "Start Test" button. You need to chose a test file for this ("Chose Test File" button, first in the top row). As an option for test you can chose which part of a test file to use (from 0 to 1.00) in a field next to "Start Test" button.

  As a result of a test you'll get a couple statistical measures (average accuracy, precision, recall, f-measure and test time).

@node SaveLoad, Research, Tests, Top
@chapter Save and Load
  You can save and load weight for the chosen number of internal layers (both matrix and graph implementations use the same weights) via "Save Weights" and "Load Weights" in the top row.

  The programm saves weights in .txt files in a specific format, wich does not allow to load weights from file fit for another number of internal layers.

@node Research, , SaveLoad, Top
@chapter Research
  As a part of project an experiment was conducted to measure time needed for different implementations to run test on a test file you can find in repository.

  Test file contains 14800 examples. The results are as follows:

@multitable {layers} {matrix} {time, s:} {xxx } {xxxx}  {1000 tests} {average}
@headitem layers @tab type @tab time, s: @tab 10 @tab 100 @tab 1000 tests @tab average
@item 5
@tab matrix
@tab 
@tab 205
@tab 2010
@tab 20446
@tab 20.415
@item 5
@tab graph
@tab 
@tab 361
@tab 3651
@tab 37523
@tab 37.418
@end multitable

@bye